\chapter{\ifenglish Introduction\else บทนำ\fi}

\section{\ifenglish Project rationale\else ที่มาของโครงงาน\fi}

% Okay there is actaully a tab down here vvv
การจัดเก็บภาษีถือเป็นเป็นหนึ่งในรายได้หลักประเทศไม่ว่าจะเป็นภาษีทางตรงอย่างเช่น ภาษีทางตรง ภาษีรายได้บุคคลธรรมดาซึ่งจะจัดเก็บได้
จากประชาชนผู้มีเงินได้ทั่วไป ภาษีเงินได้นิติบุคคลซึ่งเป็นภาษีที่จัดเก็บได้จากเงินได้ของบริษัทหรือห้างหุ้นส่วนนิติบุคคล และยังมีภาษีทางอ้อมเช่น 
ภาษีมูลค่าเพิ่ม ภาษีธุรกิจเฉพาะ ซึ่งเงินที่ได้จากการเก็บภาษีเหล่าล้วนนำไปให้รัฐบาลใช้ในการพัฒนาประเทศให้เจริญก้าวหน้า  

ภาษีป้ายก็เป็นส่วนหนึ่งของรายได้ท้องถิ่นที่สามารถจัดเก็บได้โดยองค์กรปกครองส่วนท้องถิ่น โดยที่ภาษีลักษณะนี้เมื่อจัดเก็บได้แล้ว ทางท้องถิ่น
ไม่จำเป็นต้องส่งคืนให้ทางรัฐ สามารถนำไปใช้จัดการบริหารพัฒนาภายในท้องถิ่นของตนเองได้ แต่ด้วยความสามารถในการจัดเก็บภาษีป้ายของ
องค์กรปกครองส่วนท้องถิ่นในแต่ละที่ ขึ้นอยู่กับปัจจัยหลาย ๆ อย่าง เช่น การที่ไม่สามารถรู้ได้ว่าป้ายที่สามารถจัดเก็บภาษีได้นั้นอยู่ที่ตำแหน่ง
ใดในเขตปกครอง ซึ่งมีส่วนที่ทำให้ประสิทธิภาพในการค้นหาป้ายภายในท้องถิ่นที่มีอยู่ทำได้อยู่จำกัดและเป็นขั้นตอนที่ต้องใช้กำลังคนในการตรวจสอบ
เป็นอย่างมาก ดังนั้นจากปัญหาในจุดที่กล่าวมาทำให้เกิดโครงงานที่เป็นเครื่องมือที่ช่วยในการตรวจจับหาป้ายที่คาดว่าจะสามารถนำไปจัดเก็บภาษี 
และรายงานผลให้กับแต่ละองค์กรปกครองส่วนท้องถิ่นให้ไปจัดเก็บภาษีจากป้ายเหล่านี้

\section{\ifenglish Objectives\else วัตถุประสงค์ของโครงงาน\fi}
\begin{enumerate}
    \item เพื่อสร้างระบบครบวงจรในการรับวิดีโอแล้วประมวลผลตรวจจับหาป้ายอัตโนมัติ
    \item เพื่อพัฒนาแอปพลิเคชันที่ใช้ในการอัดวิดีโอเพื่อที่จะส่งให้ระบบประมวลผล 
    \item เพื่อพัฒนาเครื่องมือในการรายงานป้ายที่ค้นพบภายในพื่นที่การปกครองส่วนท้องถิ่นสำหรับการไปจัดเก็บภาษี 
\end{enumerate}

\section{\ifenglish Project scope\else ขอบเขตของโครงงาน\fi}

\subsection{\ifenglish Hardware scope\else ขอบเขตด้านฮาร์ดแวร์\fi}

กล้องถ่ายของโทรศัพท์แต่ละเครื่องจะมีคุณภาพและลักษณ์การรถ่ายที่แตกต่างกัน ซึ่งอาจส่งผลต่อการตรวจจับวัตถุททำให้เวลานำรูปภาพที่ได้นำไป
ประมวลจะได้ผลลัพธ์ที่แตกต่างกัน ซึ่งโทรศัพท์ที่ได้ใช้ในการเก็บข้อมูลที่นำไปสร้างโมเดลมีอยู่ด้วยกัน 2 เครื่องโดยมีคุณภาพของกล้องถ่ายรูปดังนี้
\begin{itemize}
    \item Xiaomi 11T Pro ความละเอียด 108 ล้านพิกเซล 
    \item Samsung Galaxy A50s ความละเอียด 48 ล้านพิกเซล 
\end{itemize}

ความสูงของรถแต่ละคัน และมุมกล้องในการถ่ายภาพมักมีความแตกต่างกันไป ซึ่งอาจส่งผลให้ประสิทธิภาพในการตรวจจับวัตถุได้ไม่เท่ากัน 
โดยรถยนต์ที่ใช้ในการอัดวิดิโอสำหรับในการเทรนโมเดลเป็น Honda City 2024 

Mobile application ที่เป็นส่วนของการส่งข้อมูลภาพไปยังเซิฟเวอร์จำเป็นต้องเชื่อมต่ออินเทอร์เน็ตอยู่ตลอดทั้งการใช้งาน เนื่องจากต้องมีการส่งข้อมูล
ตลอดเวลา ทั้งนี้สืบก็จะมีเรื่องของการใช้งานทรัพยากรแบตเตอร์มากตามไปด้วย และในการของการแสดงผลที่เป็นเว็บแอปพลิเคชั่นจะสามารถใช้งานได้เฉพาะ
ในคอมพิวเตอร์เท่านั้น 

\subsection{\ifenglish Software scope\else ขอบเขตด้านซอฟต์แวร์\fi}

ในการเก็บภาษีป้ายนั้นจะถูกแบ่งออกเป็นป้ายหลาย ๆ ประเภท อย่างเช่น ป้ายที่มีอักษรไทยล้วน ป้ายที่มีอักษรไทยปนกับอักษรต่างประเทศหรือปนกับภาพ 
และหรือเครื่องหมา, ป้ายที่ไม่มีอักษรไทย ไม่ว่าจะมีภาพและหรือ เครื่องหมายใด ๆ ซึ่งแต่ละประเภทนั้นจะมีอัตราการเก็บภาษีที่แตกต่างกันออกไป 
แต่ในการประมวลผลในเซิฟเวอร์นั้นจะไม่มีการตรวจสอบและแบ่งแยกประเภทของป้าย และจะรวบรสมเป็นคลาสประเภทเดียวกันแทน 

\section{\ifenglish Expected outcomes\else ประโยชน์ที่ได้รับ\fi}
\begin{enumerate}
    \item ได้เครื่องมือที่ช่วยอำนวยความสะดวกในการเก็บภาษีป้ายให้มีประสิทธิภาพมากยิ่งขึ้น
\end{enumerate}

\section{\ifenglish Technology and tools\else เทคโนโลยีและเครื่องมือที่ใช้\fi}

% \subsection{\ifenglish Hardware technology\else เทคโนโลยีด้านฮาร์ดแวร์\fi}

\subsection{\ifenglish Software technology\else เทคโนโลยีด้านซอฟต์แวร์\fi}

\begin{enumerate}
    \item Visual Studio Code หรือ VSCode เป็นโปรแกรม Code Editor ที่ใช้ในการแก้ไขและปรับแต่งโค้ด จากค่ายไมโครซอฟท์ 
    มีการพัฒนาออกมาในรูปแบบของ OpenSouce ซึ่ง Visual Studio Code นั้น เหมาะสําหรับนักพัฒนาโปรแกรมที่ต้องการใช้งานข้ามแพลตฟอร์ม 
    สามารถเชื่อมต่อกับ Git ได้ นํามา ใช้งานได้ง่ายไม่ซับซ้อน มีเครื่องมือส่วนขยายต่าง ๆ ให้เลือกใช้อย่างมากมาก 

    \item Python เป็นภาษาโปรแกรมมิ่งที่มีความยืดหยุ่นสูงและสามารถนำมาใช้ในการพัฒนาโปรแกรมต่าง ๆ ได้หลากหลาย ซึ่งมีความเหมาะสมในการ
    ใช้งานในโครงการที่ต้องการประมวลผลข้อมูลที่ซับซ้อนและมีขนาดใหญ่ อย่างเช่น โมเดลการเรียนรู้เชิงลึก ที่พวกเราจะนำไปใช้กับการตรวจจับวัตถุ 
    และใช้เป็นระบบการส่งผ่านข้อมูล 
    
    \item Typescript คือภาษาคอมพิวเตอร์ที่ใช้ในการพัฒนาเว็บร่วมกับ HTML เพื่อให้เว็บมีลักษณะแบบไดนามิก หมายถึง เว็บสามารถตอบสนองกับ
    ผู้ใช้งานหรือแสดงเนื้อหาที่แตกต่างกันไปโดยจะอ้างอิงตาม เว็บบราวเซอร์ที่ผู้เข้าชมเว็บใช้งานอยู่ 
    
    \item React.js เป็นไลบรารี่จาวาสคริปที่เป็นที่ยอมรับกันว่าเป็นตัวช่วยให้สามารถสร้าง UI (User Interface หรือองค์ ประกอบของเว็บที่เชื่อมต่อ
    กับผู้ใช้งานโดยตรง) ได้แม่นยําและรวดเร็วมากยิ่งขึ้น และส่งผลให้การแสดง ผลมีความเป็นระบบคงเส้นคงวามากขึ้นไปพร้อม ๆ กัน
    
    \item YOLOv8 เป็นระบบที่ใช้ในการพัฒนาโนโมเดลตรวจจับวัตถุความเร็วสูงแบบเวลาจริง ด้วยการเรียนรู้เชิงลึกและการมองเห็นคอมพิวเตอร์ 
    
    \item Figma เครื่องมือออกแบบเว็บไซต์ แอปพลิเคชัน โลโก้ และอื่น ๆ ทําให้นักออกแบบ UX/UI สะดวก มากขึ้น ผ่านการใช้ฟี เจอร์ต่าง ๆ 
    ซึ่งมีจุดเด่นอยู่ที่การใช้งานบนได้ทุกระบบปฏิบัติการ และยังมี Community ที่ผู้ใช้สามารถแชร์ไฟล์งาน Prototype หรือ Plug-in ต่าง ๆ 
    แล้วนําไปปรับใช้กับงานของตัว เองได้ 

    \item Linux เป็นระบบปฏิบัติการ (Operating System) ที่เป็น Open Source และเป็นพื้นฐานบนหลักการของ Unix ซึ่งถูกพัฒนาขึ้นโดย 
    Linus Torvalds ในปี ค.ศ. 1991 ซึ่งเป็นระบบปฎิบัตการที่เราจะนำมาใช้งาน 
\end{enumerate}

\section{\ifenglish Project plan\else แผนการดำเนินงาน\fi}

\begin{plan}{10}{2023}{6}{2024}
    \planitem{10}{2023}{10}{2023}{เลือกอาจารย์ที่ปรึกษา และ เลือกหัวข้อโครงงาน}
    \planitem{10}{2023}{11}{2023}{ออกแบบระบบการทำงานโดยคร่าว และ เครื่องมือที่ใช้ในการทำโครงงาน}
    \planitem{11}{2023}{1}{2024}{ศึกษาข้อมูลเกี่ยวกับขอบเขตพื้นที่ที่จะใช้ทำโครงงาน}
    \planitem{2}{2024}{2}{2024}{เก็บข้อมูลเพื่อใช้ในกระบวนการเทรนโมเดลสำหรับการตรวจจับวัตถุ}
    \planitem{4}{2024}{6}{2024}{คัดเลือกข้อมูลและพัฒนาโมเดลสำหรับกระบวนการเทรนโมเดล}
    \planitem{4}{2024}{6}{2024}{ออกแบบระบบ}
\end{plan}

\begin{plan}{7}{2024}{2}{2025}
    \planitem{7}{2024}{12}{2024}{พัฒนากับทดสอบแอปพลิเคชันที่ใช้ในการอัดวิดีโอและเว็บแอปพลิเคชันในการรายงานข้อมูล}
    \planitem{1}{2025}{1}{2025}{ดิพลอยระบบโดยรวม}
    \planitem{1}{2025}{2}{2025}{ตรวจสอบความถูกต้องสมบูรณ์หลังการนำไปใช้}
    \planitem{2}{2025}{2}{2025}{เขียนรายงาน}
\end{plan}

\section{\ifenglish Roles and responsibilities\else บทบาทและความรับผิดชอบ\fi}
\begin{itemize}
    \item นาย ชาญชล ภานุศุภนิรันดร์: ทำหน้าที่ศึกษาค้นคว้าข้อมูลที่จะนํามาใช้ในโครงงาน และจัดการการเก็บข้อมูลและเชื่อม 
    ส่วนต่อประสานเชิงประยุกต์ (API: Application Programming Interface) 
    \item นาย ณัฐพงษ์ เทพพิทักษ์์: ทําหน้าที่ศึกษาค้นคว้าข้อมูลที่จะนํามาใช้ในโครงงาน และพัฒนาโมไบล์แอปพลิเคชั่นถ่ายวิดีโอสำหรับการตรวจจับวัตถุ 
    \item นาย ธนภัทร สมสิทธิ์: ทําหน้าที่ศึกษาค้นคว้าข้อมูลที่จะนํามาใช้ในโครงงาน และพัฒนาเว็บแอปพลิเคชั่นรายงานผลข้อมูลหลังจากการประมวลผลตรวจจับข้อมูล 
\end{itemize}

\section{\ifenglish%
Impacts of this project on society, health, safety, legal, and cultural issues
\else%
ผลกระทบด้านสังคม สุขภาพ ความปลอดภัย กฎหมาย และวัฒนธรรม
\fi}

การพัฒนาระบบในการตรวจจับป้ายที่สามารถนำไปเก็บภาษีได้นั้น จะช่วยอำนวยความสะดวกให้สามารถจัดการได้ง่ายและสะดวกยิ่งขึ้น 
ซึ่งมีผลกระทบในด้านกฏหมายเพราะภาษีป้ายเป็นภาษีที่จัดเก็บจากป้ายที่ แสดงชื่อ ยี่ห้อ หรือเครื่องหมายที่ใช้ในการประกอบ การค้า หรือประกอบกิจการอื่นเพื่อหารายได้ หรือ 
โฆษณาการค้า ซึ่งในส่วนของการเสียนั้นก็ขึ้นอยู่กับประเภทของป้ายตามที่กฏหมายกำหนด และรายได้ที่ได้จากการจัดเก็บภาษีก็จะถูกนำไปพัฒนาบ้านเมืองต่อไป 
