% TODO: UNCOMMENT THE FOLLOWING
% \chapter{\ifproject%
% \ifenglish Experimentation and Results\else การทดลองและผลลัพธ์\fi
% \else%
% \ifenglish System Evaluation\else การประเมินระบบ\fi
% \fi}

% TODO: DELETE THE FOLLOWING
\chapter{\ifenglish System Evaluation\else การประเมินระบบ\fi}

\ifenglish
\else
\section{การประเมินประสิทธิภาพซอฟต์แวร์}
ทดสอบประสิทธิภาพซอฟต์แวร์โดยจะมีการแบ่งส่วนในการทดสอบออกเป็นส่วน ๆ เพื่อให้รู้ว่าในแต่ละส่วนของซอฟต์แวร์ของเรานั้น 
ทำงานได้อย่างมีประสิทธิภาพหรือไม่ จึงสามารถแบ่งออกการประเมินได้เป็นดังนี้ 
\begin{enumerate}
    \item Classification model - เป็นการทดสอบเพื่อประเมินและตรวจสอบความเร็วในการประมวลผลเพื่อทำการ classify 
    ว่า object ใดเป็นป้ายที่สามารถจัดเก็บภาษีได้ รวมถึงในเรื่องของความแม่นยำในการ classify  
    \item Response time - เป็นการทดสอบเพื่อประเมินในเรื่องของความเร็วในการรับส่งข้อมูลระหว่าง client กับ application server  
\end{enumerate}

\section{การประเมินความพึงพอใจในการใช้งานระบบ}
ทดสอบความพึงพอใจในการใช้งานจะมีการแบ่งออกเป็นสองส่วน คือส่วนของแอปพลิเคชันในโทรศัพท์มือถือ กับส่วนของเว็บแอปพลิเคชัน 
โดยจะมีเกณฑ์การให้คะแนนอยู่ที่ 1 ถึง 5 โดยจะมีการให้คะแนนในเรื่องดังต่อไปนี้ 
\begin{enumerate}
    \item ความง่ายต่อการใช้งานของแอปพลิเคชัน 
    \item ความสะดวกในการใช้งานในตอนเริ่มต้นของแอปพลิเคชัน 
    \item ความดึงดูดในการใช้งานของแอปพลิเคชัน        
    \item ประโยชน์ที่มีของแอปพลิเคชัน 
\end{enumerate}

โดยที่ทั้ง 4 ข้อเป็นพิจราณาจากแนวคิดตาม The Four Elements of User Experience \cite{uxquantification} ที่ประกอบไป ด้วย 

\begin{enumerate}
    \item Usability ความใช้ง่ายในการใช้งาน เกี่ยวข้องกับสามารถในการใช้งาน รวมไปถึงความเหมาะสมการใช้งานกับผู้งานใช้ 
    \item Adaptability ความสามารถในงานปรับตัว กล่าวถึงระดับความยากง่ายของการใช้งานตั้งแต่จุดเริ่มต้น จนถึงจุดสิ้นสุดของระบบ โดยที่ผู้งานสามารถใช้งานได้อย่างคล่องแคล่ว 
    \item Desirability ความพึงพอใจ คือเมื่อใช้งานแล้วผู้ได้รับประสบการณ์ที่ดีในจากใช้งานของระบบ 
    \item Value คุณค่าของระบบ คือระบบที่ผู้ใช้เข้ามาใช้งานมีความสอดคล้องกับความต้องการของผู้ใช้ 
\end{enumerate}

โดยในส่วนของการประเมินความพึงพอใจในระบบทั้งในส่วนของเว็บไซต์และในส่วนของมือถือ ได้มีการเก็บผลการประเมินมาจากกลุ่มผู้ทดลองการใช้งานทั้ง 15 คน ซึ่งจะมีผลการประเมินเฉลี่ยเป็นดังรูป \ref{fig:evaluation-th}
\fi

\clearpage

\ifenglish
\insertPDFfigure{resources/eval/Average-Website-&-Mobile-Evaluation-by-Aspect.pdf}{Evaluation of System Satisfaction}{evaluation}
\else
\insertPDFfigure{resources/eval/Average-Website-&-Mobile-Evaluation-by-Aspect-thai.pdf}{ผลการประเมินความพึงพอใจในการใช้งานระบบ}{evaluation-th}

จากข้อมูลที่สามารถเห็นได้จากรูปภาพข้างต้น จะเห็นได้ว่าความพึงพอใจในการใช้งานระบบของโดยรวมในทุกด้านั้นมีค่าเฉลี่ยอยู่ที่ 4 ไปจนถึง 4.45 ซึ่งเป็นค่าที่สูงมาก แสดงให้เห็นว่าผู้ใช้งานมีความพึงพอใจในการใช้งานระบบของเราอย่างมาก โดยเราสามารถที่จะวิเคราะห์ในแต่ละด้านได้ดังนี้

\begin{enumerate}
    \item ระบบนี้มีประโยชน์ จากการประเมินของกลุ่มผู้ทดสอบทำให้เราทราบได้ว่าคะแนนของในส่วนเว็บไซต์กับในส่วนของมือถือนั้นจะมีค่าที่ใกล้เคียงกันอยู่ที่ประมาณ 4.3 เป็นการบ่งบอกได้ว่าระบบนี้มีประโยชน์ต่อสายตาของผู้ใช้งาน
    \item ระบบนี้ใช้งานง่าย ความง่ายของการใช้งานที่กลุ่มผู้ทดสอบได้ทดลองใช้ทั้งในส่วนของเว็บไซต์และในส่วนของมือถือ มีค่าเฉลี่ยอยู่ที่ 4.4 ซึ่งเป็นค่าที่สูงมาก แสดงให้เห็นว่าระบบนี้มีความง่ายในการใช้งานสำหรับผู้ที่ใช้งานที่ได้ทดลองใช้งานเป็นครั้งแรก
    \item ระบบนี้น่าเอาไปใช้ ความน่าไปใช้งานของระบบที่ได้รับคะแนนจากกลุ่มผู้ทดสอบที่ใช้งานทั้งในส่วนของเว็บไซต์และในส่วนของมือถือ มีค่าเฉลี่ยอยู่ที่ 4.4 ซึ่งเป็นค่าที่สูงมาก แสดงให้เห็นว่าระบบนี้น่าไปใช้งานสำหรับผู้ที่เข้ามาทดลองใช้งาน
    \item ระบบมีความสนุก ความสนุกในการใช้งานระบบที่ได้รับคะแนนจากกลุ่มผู้ทดสอบที่ใช้งานทั้งใน\\ ส่วนของเว็บไซต์และในส่วนของมือถือ มีค่าเฉลี่ยอยู่ที่ 4 ซึ่งเป็นการบ่งบอกว่าระบบนี้มีความสนุกในการใช้งานที่น้อยกว่าในด้านที่เหลือ ควรที่จะมีการปรับปรุงให้ดียิ่งขึ้น เพื่อให้ผู้ที่เข้ามาใช้งานนั้นรู้สึกสนุกเมื่อเข้ามาใช้งาน
\end{enumerate}
\fi