\subsection{\ifenglish Object Association Process\else การดำเนินการเชื่อมโยงวัตถุ\fi}
\ifenglish
The Object Association Process groups identical objects across multiple frames and assigns them relevant attributes. To achieve this, the system maintains a mapping of object IDs to the frames in which they appear. When an object is not detected for a predefined number of consecutive frames, the process assumes that the recorder has moved past the object. At this point, the mapping of the object ID to its associated frames is used to determine its final location.

The object's location is inferred from the last detected frame by correlating its timestamp with the closest recorded location timestamp. Given that timestamps are stored alongside location coordinates, the location of the object is determined by finding the closest matching timestamp in the location data. Additionally, the image representing the object in the system's interface is chosen from the third quartile of its detected frames. This selection ensures that the image is sufficiently large for visibility while avoiding frames that might have the object partially out of view.

Finally, the processed data is sent to the queue in the following structured format:
\else
การดำเนินการเชื่อมโยงวัตถุจะทำการจัดกลุ่มวัตถุที่เป็นวัตถุเดียวกันในหลายแต่ละเฟรมและกำหนดข้อมูลที่เกี่ยวข้องให้กับวัตถุเหล่านั้น โดยที่ระบบจะเก็บแมปของรหัสวัตถุกับลิสท์ของทุกเฟรมที่วัตถุนั้นปรากฏ เมื่อไม่พบวัตถุติดต่อกันเป็นจำนวนเฟรมที่กำหนดไว้ล่วงหน้า กระบวนการเชื่อมโยงจะถือว่าผู้บันทึกได้ขับผ่านวัตถุนั้นไปแล้ว การแมปรหัสวัตถุกับเฟรมที่เกี่ยวข้องในจุดนี้จะถูกใช้เพื่อกำหนดตำแหน่งสุดท้ายของวัตถุ

ตำแหน่งของวัตถุจะถูกอนุมานจากเฟรมสุดท้ายที่ตรวจพบด้วยการเชื่อมโยงเวลาการบันทึกเฟรมกับตำแหน่งที่มีเวลาบันทึกใกล้เคียงที่สุด นอกจากนี้ ภาพที่แสดงวัตถุบนอินเตอร์เฟซของระบบจะถูกเลือกจากควอไทล์ที่สามของเฟรมที่ตรวจพบทั้งหมด เพื่อทำให้มั่นใจได้ว่าวัตถุในเฟรมมีขนาดใหญ่พอสำหรับการมองเห็น ในขณะที่วัตถุไม่ออกนอกเฟรมมากเกินไป

สุดท้าย ข้อมูลที่ประมวลผลจะถูกส่งไปยังคิวในรูปแบบที่มีโครงสร้างดังนี้:
\fi
\begin{lstlisting}
{
    "frame": object,
    "tloc": {
        "timestamp": long,
        "latitute": float,
        "longitude": float
    },
    "recordedAt": long
}
\end{lstlisting}

\insertPDFfigure
{resources/detection/detection-association-process.pdf}
{\ifenglish Tracking Association Workflow\else ขั้นตอนการทำงานของการดำเนินการเชื่อมโยงวัตถุ\fi}
{tracking association workflow}