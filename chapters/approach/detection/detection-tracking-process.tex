\subsection{\ifenglish Object Tracking Process\else การดำเนินการติดตามวัตถุ\fi}
\ifenglish
The Tracking Process focuses on performing object detection and tracking across video frames using Ultralytics' YOLO model. The process is designed to identify objects in each frame and assign a unique tracking ID to each object. This ensures consistent tracking of objects across multiple frames. The output from this process includes several key components: the detected frame object from YOLO, the Unix timestamp indicating when the frame was recorded, and a list of detected objects. Each object is accompanied by its tracking ID and bounding box coordinates, allowing for precise identification and localization within the frame.
\else
การดำเนินการติดตามวัตถุเน้นการตรวจจับและติดตามวัตถุในเฟรมวิดีโอโดยใช้โมเดล YOLO ของ Ultralytics กระบวนการนี้ออกแบบมาเพื่อระบุวัตถุในแต่ละเฟรมและกำหนดรหัสติดตามที่ไม่ซ้ำกันให้กับแต่ละวัตถุ ซึ่งจะช่วยให้การติดตามวัตถุในหลายเฟรมเป็นไปอย่างต่อเนื่อง ผลลัพธ์จากกระบวนการนี้ประกอบด้วยส่วนประกอบสำคัญหลายประการ ได้แก่ วัตถุที่ตรวจจับได้จาก YOLO, Unix timestamp ที่ระบุเวลาที่บันทึกเฟรม และรายการวัตถุที่ตรวจจับได้ แต่ละวัตถุจะมีรหัสติดตามและพิกัดกรอบล้อมรอบเพื่อให้สามารถระบุและระบุตำแหน่งได้อย่างแม่นยำในเฟรม
\fi
\begin{lstlisting}
{
    "frame": object,
    "recordedAt": long,
    "trackingBoxes": {
        "trackId": number, 
        "box": (int, int, int, int) 
    }[],
}
\end{lstlisting}

\insertPDFfigure
{resources/detection/detection-tracking-process.pdf}
{\ifenglish Tracking Process Workflow\else ขั้นตอนการทำงานของการดำเนินการติดตามวัตถุ\fi}
{tracking process workflow}