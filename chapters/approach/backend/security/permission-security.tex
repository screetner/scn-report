\subsubsection{Permission Security}

\ifenglish
\else
เป็นการอธิบายถึงการตรวจสอบสิทธิ์ของผู้ใช้งานในการเข้าถึง API ระบบจะตรวจสอบสิทธิ์ของผู้ใช้งานทุกครั้งที่มีการเข้าถึง API เนื่องจากแต่ละ API นั้นได้มีการกำหนดสิทธิ์ในการเข้าถึงที่แตกต่างกัน โดยระบบจะใช้ข้อมูลของผู้ใช้งานและสิทธิ์ของผู้ใช้งานที่ถูกกำหนดไว้ในระบบเพื่อตรวจสอบสิทธิ์ของผู้ใช้งานทุกครั้งที่มีการเข้าถึง API

Token จะถูกสร้างขึ้นจากข้อมูลที่เป็นเฉพาะตัวของผู้ใช้งานและถูกเก็บไว้ในระบบเพื่อใช้ในการยืนยันตัวตนในครั้งถัด ๆ ไป นอกจากนี้ ระบบยังมีการตรวจสอบสิทธิ์จากข้อมูลที่ถูกเก็บไว้ใน Redis ข้อมูลที่เก็บไว้ใน Redis ประกอบด้วยข้อมูลของผู้ใช้งานและสิทธิ์ของผู้ใช้งานที่ถูกกำหนดไว้ในระบบ เพื่อให้การตรวจสอบสิทธิ์ของผู้ใช้งานเป็นปัจจุบันมากที่สุด โดยจะมีการอธิบายดังต่อไปนี้

\paragraph{ข้อมูลที่ใช้ในการตรวจสอบสิทธิ์} การตรวจสอบสิทธิ์ในการเข้าถึง API ใช้ข้อมูลดังต่อไปนี้:
\begin{lstlisting}
    {
        "mobile": {
            "access": boolean,
            "videosProcess": boolean,
        },
        "web": {
            "access": boolean,
            "manageGeometry": boolean,
            "member": {
                "invite": boolean,
            },
            "role": {
                "create": boolean,
                "delete": boolean,
                "managePermission": boolean,
                "manageMember": boolean,
            },
        },
    }
\end{lstlisting}
ซึ่งจะเป็นในรูปแบบเดียวกับ rolePermission ที่เคยกล่าวถึงไปก่อนหน้านี้

\paragraph{Redis}
ข้อมูลที่ถูกเก็บไว้ใน Redis ประกอบด้วยข้อมูลของผู้ใช้งานและสิทธิ์ของผู้ใช้งาน โดยมีการเก็บข้อมูลในรูปแบบดังนี้:
\begin{itemize}
    \item \textbf{user:role}: เก็บข้อมูลการเชื่อมโยงระหว่างผู้ใช้งาน (userId) และบทบาท (roleId) ของผู้ใช้งาน
    \item \textbf{role:permission}: เก็บข้อมูลสิทธิ์ของบทบาท (roleId) ในรูปแบบ JSON
\end{itemize}

ตัวอย่างการเก็บข้อมูลใน Redis:
\begin{lstlisting}
    redisClient.hset('user:role', userId, roleId)
    redisClient.hset('role:permission', roleId, JSON.stringify(permission))
\end{lstlisting}

การเก็บข้อมูลใน Redis ช่วยให้ระบบสามารถตรวจสอบสิทธิ์ของผู้ใช้งานได้อย่างรวดเร็วและมีประสิทธิภาพ เนื่องจาก Redis เป็นฐานข้อมูลที่ทำงานในหน่วยความจำ (in-memory database) ซึ่งสามารถเข้าถึงข้อมูลได้รวดเร็วกว่าฐานข้อมูลแบบดั้งเดิม
\fi