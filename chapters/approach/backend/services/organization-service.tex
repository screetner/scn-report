\subsubsection{Organization Service}

\ifenglish
\else
บริการองค์กร (Organization Service) ใช้สำหรับจัดการข้อมูลขององค์กรต่างๆ ที่ใช้สำหรับคนที่เป็นเจ้าของเว็บไซต์

ฟังก์ชันใน Organization Service ประกอบไปด้วย

\begin{itemize}
    \item \textbf{getAllOrganization(): Promise<organizationData[]>}: ดึงข้อมูลขององค์กรทั้งหมด
    \item \textbf{createOrganization(name: string): Promise<createOrganizationResult>}: สร้าองค์กรใหม่ขึ้นมา
    \item \textbf{getOrganizationInformation(orgId: string): Promise<organizationInformation>}: ดึงข้อมูลโดยละเอียดขององค์กรนั้นมา
    \item \textbf{getInviteList(orgId: string): Promise<inviteData[]>}:\\ ดึงข้อมูลของคำเชิญทั้งหมดขององค์กรนั้น
    \item \textbf{getSAS(orgId: string): Promise<string>}: ดึง SAS Token ขององค์กรนั้น
\end{itemize}

ข้อมูลใน organizationData ถูกจัดเก็บในรูปแบบดังนี้:
\begin{lstlisting}
    {
        "orgId": string,
        "orgName": string,
        "orgMember": number,
        "orgAssets": number
    }
\end{lstlisting}

ข้อมูลใน createOrganizationResult ถูกจัดเก็บในรูปแบบดังนี้:
\begin{lstlisting}
    {
        "orgId": string,
        "defaultRoleId": string,
        "adminRoleId": string
    }
\end{lstlisting}

\clearpage

ข้อมูลใน organizationInformation ถูกจัดเก็บในรูปแบบดังนี้:
\begin{lstlisting}
    {
        "name": string,
        "border": borderData[] | null,
        "createdAt": Date | null,
        "updatedAt": Date | null
    }
\end{lstlisting}

ข้อมูลใน borderData ถูกจัดเก็บในรูปแบบดังนี้:
\begin{lstlisting}
    {
        "latitude": number,
        "longitude": number
    }
\end{lstlisting}

ข้อมูลใน inviteData ถูกจัดเก็บในรูปแบบดังนี้:
\begin{lstlisting}
    {
        "inviterEmail": string | null,
        "inviteeEmail": string,
        "time": Date | null
    }
\end{lstlisting}
\fi