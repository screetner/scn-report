\subsubsection{VideoSession Service}

\ifenglish
\else
บริการ VideoSession นั้นใช้สำหรับการจัดการวิดีโอที่กำลังถูกอัพโหลดจากผู้ใช้งาน และรวมถึงมีส่วนที่ถูกใช้งานสำหรับการประมวลผลรูปภาพด้วย

ฟังก์ชันใน VideoSession Service ประกอบไปด้วย
\begin{itemize}
    \item \textbf{postVideoSession(
        userId: string,
        postData: {
          uploadProgress: number
          videoNames: string[]
          videoSessionName: string
        },
    ): Promise<\{videoSessionId: string\}>}:
    ฟังก์ชันนี้ใช้สำหรับการสร้าง VideoSession ใหม่ โดยจะมีชุดของ video ที่อยู่ใน session นี้และส่งค่า videoSessionId กลับไปให้ผู้ใช้งาน
    \item \textbf{updateVideoSession(
        videoSessionId: string,
        uploadProgress: number,
        \\userId: string,
    ): Promise<boolean>}: ฟังก์ชันนี้ใช้สำหรับการอัพเดทค่า uploadProgress ของ videoSession ที่กำลังอัพโหลดอยู่
    \item \textbf{updateVideoSessionHelper(
        videoSessionId: string,
        uploadProgress: number,
        \\state: videoSessionStateEnum,
    ): Promise<\{videoSessionId: string\}[]>}: ฟังก์ชันนี้ใช้สำหรับการอัพเดทค่า uploadProgress และ state ของ videoSession ที่กำลังอัพโหลดอยู่ ซึ่งจะถูกเรียกโดยฟังก์ชันข้างต้น เพราะว่าในส่วนนี้จะถูกนำไปใช้ที่อื่นอยู่ จึงมีการแยกส่วนนี้ออกมาเป็นฟังก์ชันใหม่
    \item \textbf{removeVideoSession(videoSessionId: string): Promise<\{videoSessionId: string\}>}: ฟังก์ชันนี้ใช้สำหรับการลบ videoSession ที่สามารถลบได้เท่านั้น
    \item \textbf{getVideoSession(videoSessionId: string): Promise<\{videoSessionName: string\}>}: ฟังก์ชันนี้ใช้สำหรับการดึงข้อมูล videoSession จากฐานข้อมูล
    \item \textbf{generateSessionFolderPath(
        data: GenerateVideoSession,
      ): \\Promise<RequestPythonDetection>}:
    \\ฟังก์ชันนี้ใช้สำหรับการสร้างสิ่งที่จำเป็นในการทำงานของ python ในการประมวลผลวิดีโอ
\end{itemize}

ข้อมูลใน videoSessionStateEnum ประกอบไปด้วย
\begin{lstlisting}
    {
        UPLOADING = 'uploading',
        PROCESSING = 'processing',
        CAN_DELETE = 'canDelete',
    }
\end{lstlisting}

ข้อมูลใน GenerateVideoSession ถูกจัดเก็บในรูปแบบดังนี้:
\begin{lstlisting}
    {
        "videoSessionId": string,
        "organizationId": string[],
        "organizationName": string
    }
\end{lstlisting}

\clearpage

ข้อมูลใน RequestPythonDetection ถูกจัดเก็บในรูปแบบดังนี้:
\begin{lstlisting}
    {
        "sessionId": string,
        "session_folder_path": string[],
        "session_detected_folder_path": string
    }
\end{lstlisting}
\fi