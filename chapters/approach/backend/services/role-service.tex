\subsubsection{Role Service}

\ifenglish
\else
บริการบทบาท (Role Service) ใช้สำหรับจัดการกับบทบาทของผู้ใช้งานในระบบ 

ฟังก์ชันใน Role Service ประกอบด้วย
\begin{itemize}
    \item \textbf{getRoleOrganization(
        organizationId: string
      ): Promise<roleInOrg[]>}: สำหรับทำการดึงข้อมูลบทบาทขั้นต้นขององค์กรออกมา
    \item \textbf{getUnassignedRole(
        organizationId: string
      ): \\Promise<roleMemberInformation[]>}: สำหรับทำการดึงข้อมูลของสมาชิกภายในองค์กรที่ยังไม่มีบทบาทเป็นของตัวเองออกมา
    \item \textbf{getRoleInformation(
        roleId: string,
        organizationId: string
      ):\\Promise<roleManagement>}: สำหรับทำการดึงข้อมูลของบทบาทที่เลือกออกมา ซึ่งจะรวมถึงข้อมูลเบื้งต้นของสมาชิกที่มีบทบาทดังกล่าวด้วย
    \item \textbf{assignRole(
        userIds: string[],
        roleId: string,
        organizationId: string
      ): \\Promise<updateRoleUser[]>}: สำหรับทำการกำหนดบทบาทให้กับสมาชิกที่เลือก
    \item \textbf{unassignRole(
        userId: string,
        organizationId: string
      ): \\Promise<updateRoleUser>}: สำหรับทำการลบบทบาทของสมาชิกที่เลือก
    \item \textbf{changeRoleName(
        roleId: string,
        roleName: string,
        organizationId: string
      ): \\Promise<updateRoleName>}: สำหรับทำการเปลี่ยนชื่อของบทบาทที่เลือก
    \item \textbf{deleteRole(
        roleId: string,
        organizationId: string
      ): Promise<roleInformation>}: สำหรับทำการลบบทบาทที่เลือก
    \item \textbf{createRole(
        organizationId: string
      ): Promise<roleInformation>}: สำหรับทำการสร้างบทบาทใหม่
    \item \textbf{updateRolePermission(
        roleId: string,
        permission: rolePermission,\\
        organizationId: string
      ): Promise<roleInformation>}: สำหรับทำการปรับเปลี่ยนสิทธิ์ของบทบาทที่เลือก
    \item \textbf{getAdminId(organizationId: string): Promise<string>}: สำหรับทำการดึงรหัสบทบาทของผู้ดูแลระบบขององค์กร
\end{itemize}

ข้อมูลใน roleInOrg ถูกจัดเก็บในรูปแบบดังนี้:
\begin{lstlisting}
    {
        "roleId": string,
        "roleName": string,
        "members": number,
    }
\end{lstlisting}

\clearpage

ข้อมูลใน roleMemberInformation ถูกจัดเก็บในรูปแบบดังนี้:
\begin{lstlisting}
    {
        "userId": string,
        "username": string,
        "email": string,
    }
\end{lstlisting}

ข้อมูลใน roleInformation ถูกจัดเก็บในรูปแบบดังนี้:
\begin{lstlisting}
    {
        "roleId": string,
        "roleName": string,
    }
\end{lstlisting}

ข้อมูลใน rolePermission ถูกจัดเก็บในรูปแบบดังนี้:
\begin{lstlisting}
    {
        "mobile": {
            "access": boolean,
            "videosProcess": boolean,
        },
        "web": {
            "access": boolean,
            "manageGeometry": boolean,
            "member": {
                "invite": boolean,
            },
            "role": {
                "create": boolean,
                "delete": boolean,
                "managePermission": boolean,
                "manageMember": boolean,
            },
        },
    }
\end{lstlisting}

ข้อมูลใน roleManagement ถูกจัดเก็บในรูปแบบดังนี้:
\begin{lstlisting}
    {
        "roleInfo": roleInformation,
        "roleMembers": roleMemberInformation[],
        "rolePermissions": rolePermission,
    }
\end{lstlisting}

ข้อมูลใน updateRoleUser ถูกจัดเก็บในรูปแบบดังนี้:
\begin{lstlisting}
    {
        "userId": string,
    }
\end{lstlisting}

ข้อมูลใน updateRoleName ถูกจัดเก็บในรูปแบบดังนี้:
\begin{lstlisting}
    {
        "roleId": string,
        "oldName": string,
        "newName": string,
    }
\end{lstlisting}
\fi