\subsubsection{หน้าหลัก}
\ifenglish \else
หน้านี้จะแสดงแผนที่พร้อมทั้งขอบเขตขององค์กรที่ได้มีการตั้งค่าไว้ และรวมไปถึงตำแหน่งของป้ายโฆษณาที่ระบบตรวจจับได้จากการบันทึกจากแอปพลิเคชัน โดยการแสดงผลจะแบ่งเป็น 2 ชั้น คือ
\begin{itemize}
    \item heatmap แสดงความถี่ของการตรวจจับป้ายโฆษณาในพื้นที่นั้น ๆ
    \item ตำแหน่งของป้ายโฆษณาที่ตรวจจับได้จากการบันทึกจากแอปพลิเคชัน
\end{itemize}
โดยการแสดงผลทั้งสองชั้นนั้นจะมีเงื่อนไขการแสดงผลที่แตกต่างกันคือ หากระยะการ zoom แผนที่ของผู้ใช้งานมีค่าลดลงถึงจุด ๆ หนึ่งจะเปลี่ยนแปลงการแสดงผลให้เป็นแบบ ระบุตำแหน่ง เช่นเดียวกันหากผู้ใช้งานมีการ zoom แผนที่ให้มีค่าเพิ่มขึ้นถึงจุด ๆ หนึ่งจะเปลี่ยนแปลงการแสดงผลให้เป็นแบบ heatmap แทนเพื่อแสดงให้ผู้ใช้เห็นภาพรวมของการตรวจจับป้ายโฆษณาในพื้นที่นั้น ๆ ได้ดีขึ้น ดังรูปที่ \ref{fig:org-dash}

หากผู้ใช้งานอยากทราบว่า ณ ตำแหน่งที่ระบุบนแผนที่นั้นมีป้ายอยู่จริงหรือไม่ สามารถคลิกที่ตำแหน่งนั้น ๆ บนแผนที่เพื่อดูข้อมูลเพิ่มเติมได้ โดยจะแสดงข้อมูลดังนี้
\begin{itemize}
    \item รูปภาพป้ายโฆษณาที่ตรวจจับได้พร้อมกับการตีกรอบไว้เพื่อบอกว่าเป็นป้ายอันใดในรูป
    \item ชื่อของผู้ใช้งานที่บันทึกข้อมูล
    \item ประเภทของ Asset ที่บันทึกข้อมูล
    \item ปุ่มกดเพื่อไปยังหน้า Google Map เพื่อนำทางไปยังตำแหน่งนั้น ๆ
    \item \textbf{ปุ่มกดลบรูปภาพในตำแหน่งนั้น ๆ หากระบบตรวจจับผิดพลาด}
\end{itemize}
โดยที่หลังจากกดที่ตำแหน่งที่ต้องการดูข้อมูลเพิ่มเติมแล้ว ระบบจะแสดงข้อมูลดังกล่าวในหน้าต่างใหม่ ดังรูปที่ \ref{fig:org-dash-loc}
\fi

\insertPDFfigure{resources/frontend/org-dash.png}{\ifenglish Home page \else หน้าหลักของผู้ใช้งานทั่วไป \fi}{org-dash}
\insertPDFfigure{resources/frontend/org-dash-loc.png}{\ifenglish More deatil of that location \else หน้าแสดงข้อมูลของป้าย  \fi}{org-dash-loc}