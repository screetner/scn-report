\subsubsection{\ifenglish Role Management \else การจัดการบทบาท \fi}
\ifenglish \else
การจัดการบทบาท (Role Management) เป็นกระบวนการที่ผู้ดูแลระบบมีสิทธิ์ในการกำหนดและควบคุมบทบาทของผู้ใช้งานภายในองค์กรอย่างมีประสิทธิภาพ โดยผู้ดูแลระบบสามารถดำเนินการเพิ่ม ลบ หรือแก้ไขบทบาทที่มีอยู่ได้อย่างอิสระ เพื่อให้สอดคล้องกับความต้องการและนโยบายขององค์กร

ผู้ใช้งานแต่ละคนจะสามารถอยู่ในบทบาทใดบทบาทหนึ่งเท่านั้น โดยบทบาทดังกล่าวจะกำหนดสิทธิ์การเข้าถึงและใช้งานฟังก์ชันต่าง ๆ ของระบบตามที่ผู้ดูแลระบบได้กำหนดไว้ ซึ่งสิทธิ์ในการใช้งานนี้จะถูกกำหนดอย่างชัดเจนเพื่อป้องกันการเข้าถึงข้อมูลหรือฟังก์ชันที่ไม่เหมาะสมกับระดับบทบาทของผู้ใช้งาน

ทั้งนี้ ผู้ใช้งานจะไม่สามารถเปลี่ยนแปลงบทบาทของตนเองได้ หากมีความจำเป็นต้องเปลี่ยนแปลงบทบาทเพื่อให้สามารถเข้าถึงสิทธิ์การใช้งานในระดับที่สูงขึ้น จะต้องยื่นคำร้องขอเปลี่ยนแปลงบทบาทต่อผู้ดูแลระบบ เพื่อให้ผู้ดูแลระบบทำการพิจารณาและอนุมัติการเปลี่ยนแปลงดังกล่าวตามกระบวนการที่กำหนดไว้

กระบวนการจัดการบทบาทนี้มีความสำคัญในการรักษาความปลอดภัยของระบบ รวมถึงช่วยให้การบริหารจัดการสิทธิ์การเข้าถึงข้อมูลและฟังก์ชันต่าง ๆ เป็นไปอย่างมีประสิทธิภาพและสอดคล้องกับหลักการกำกับดูแลขององค์กร

ในส่วนแรกหลังจากกดเข้ามาที่เมนู "จัดการบทบาท" ผู้ใช้งานจะเห็นตารางที่แสดงดังรูป \ref{fig:org-tole-table} ซึ่งแสดงรายละเอียดของบทบาททั้งหมดที่มีในองค์กร ไม่ว่าจะเป็นชื่อบทบาท และจำนวนคนที่อยู่ในบทบาทนั้น ๆ

จากรูปผู้ดูแลระบบสามารถที่จะเพิ่มบทบาทใหม่ได้โดยการกดปุ่ม Create Role ที่ด้านบนขวาของตาราง ระบบจะสร้างบทบาทใหม่ขึ้นมาโดยการใช้ชื่อเป็นชื่อที่ระบบตั้งให้  และในท้ายที่สุดผู้ดูแลระบบจะต้องเข้าไปแก้ไขบทบาทนั้นเพื่อเปลี่ยนชื่อและกำหนดสิทธิ์การใช้งานของบทบาทนั้น ๆ ต่อไป
\fi

\insertPDFfigure{resources/frontend/org-role-table.png}{\ifenglish All role in organization \else แสดงบทบาททั้งหมดในองค์กร \fi}{org-tole-table}

\clearpage

\ifenglish \else
ในส่วนต่อมาหากกดเข้าไปที่ชื่อของบทบาทนั้น ๆ ในตารางที่แสดงในรูป \ref{fig:org-tole-table} หรือการกดปุ่มเพื่อสร้างบทบาทใหม่ ก็จะเข้าสู่หน้าของการตั้งค่าบทบาทนั้น ๆ ตามรูป \ref{fig:org-role-mng} ซึ่งในรายละเอียดของภาพดังกล่าวจะประกอบไปด้วยส่วนต่าง ๆ ดังนี้
\begin{itemize}
    \item ทางซ้ายมือจะเป็นส่วนที่บอกว่ามีบทบาทอะไรบ้างที่สามารถตั้งค่าได้ กล่าวคือเป็นบทบาททั้งหมดที่มีในระบบขององค์กรที่ผู้ใช้งานสามารถเข้าถึงได้ สามารถคลิกเพื่อเข้าไปตั้งค่าบทบาทนั้น ๆ ได้ (ไม่จำเป็ฯต้องย้อนกลับไปหน้าก่อนหน้าที่กดเข้ามา)
    \item ส่วนที่เหลือจะเป็นส่วนที่สามารถตั้งค่าบทบาทนั้น ๆ ได้โดยจากรูป \ref{fig:org-role-mng} จะเป็นการแก้ไขเปลี่ยนแปลงชื่อบทบาท
    \item ส่วนของการตั้งต่าสิทธิ์การใช้งานของบทบาทนั้น ๆ ตามรูป \ref{fig:org-role-mng2} โดยผู้ที่มีสิทธิ์ในการแก้ไขสามารถทำได้สามารถปรับแต่งได้อย่างอิสระตามความต้องการ โดยการกดปุ่มตามสิทธ์ที่ต้องการให้ใช้งานได้หรือไม่ให้ใช้งานได้
    \item ในส่วนการตั้งค่าสุดท้ายคือการกดหนดว่าผู็ใช้งานคนใดจะอยู่ในบทบาทนี้ ตามรูป \ref{fig:org-role-mng3} โดยจะแสดงตารางเพื่อระบุว่าขณะนั้นมีผู้ใช้งานในบทบาทนี้อยู่กี่คน และสามารถเพิ่มหรือลบผู้ใช้งานได้
\end{itemize}
โดยมนรายละเอียดการของจัดการแต่ละส่วนได้มีการอธิบายไว้ในส่วนของการทำงานของระบบหลังบ้านแล้วท่านสามารถอ่านเพิ่มเติมได้จากส่วนนั้น
\fi

\insertPDFfigure{resources/frontend/org-role-mng.png}{\ifenglish Role settings I \else ตั้งค่าบทบาท 1 \fi}{org-role-mng}
\insertPDFfigure{resources/frontend/org-role-mng2.png}{\ifenglish Role settings II \else ตั้งค่าบทบาท 2 \fi}{org-role-mng2}
\insertPDFfigure{resources/frontend/org-role-mng3.png}{\ifenglish Role settings III \else ตั้งค่าบทบาท 3 \fi}{org-role-mng3}