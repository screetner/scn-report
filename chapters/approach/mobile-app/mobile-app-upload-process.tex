\subsection{\ifenglish Video Session Upload Process \else กระบวนการอัปโหลดเซสชันวิดีโอ \fi}

\ifenglish
When the user chooses to upload a video session, the Directory Upload Manager initializes a background process using the WorkManager, which handles the upload process in a headless manner as shown in \autoref{fig:workmanager instantiation}. The upload manager ensures that all necessary application context is provided to manage the processes, including the file upload process handled by the Directory Upload Client and the device notification process managed by the Notification Manager.
\else
เมื่อผู้ใช้เลือกที่จะทำการอัปโหลดเซสชันวิดีโอ ตัว Directory Upload Manager จะเริ่มกระบวนการเบื้องหลังโดยใช้ WorkManager ซึ่งจัดการการอัปโหลดโดยขาดส่วนติดต่อกับผู้ใช้โดยตรง Directory Upload Manager ตาม \autoref{fig:workmanager instantiation} จึงต้อง ทำให้มั่นใจว่ามีการจัดเตรียมคอนเท็กซ์ของแอปพลิเคชันที่จำเป็นทั้งหมดก่อนที่จะสร้าง WorkManager รวมถึงการเริ่มต้นกระบวนการอัปโหลดไฟล์ที่ดำเนินการโดย Directory Upload Client และกระบวนการแจ้งเตือนอุปกรณ์ที่จัดการโดย Notification Manager
\fi

\insertPDFfigure[page=1,width=100mm]
{resources/mobile-app/workmanager-instantiation.pdf}
{\ifenglish Upload Process Task Instantiation \else กระบวนการสร้างงานสำหรับการอัปโหลด \fi}
{workmanager instantiation}

\ifenglish
The directory upload process utilizes a resumable upload mechanism. The underlying technology used is the Tus protocol, which involves a Tus server on the receiving end. This server handles incoming file packets from the mobile application and routes them into Azure Blob Storage. The mobile application streams the file in smaller chunks and tracks the upload progress locally. This process is managed by the Tus client, as shown in \autoref{fig:file upload sequence}.
\else
กระบวนการอัปโหลดไดเรกทอรีใช้กลไกการอัปโหลดที่สามารถกลับมาอัปโหลดต่อได้ เทคโนโลยีพื้นฐานที่ใช้คือโปรโตคอล Tus ซึ่งต้องมีเซิร์ฟเวอร์ Tus ฝั่งรับ เซิร์ฟเวอร์นี้จะจัดการแพ็กเก็ตไฟล์ที่ส่งเข้ามาจากแอปพลิเคชันมือถือและส่งไปยัง Azure Blob Storage โดยแอปพลิเคชันมือถือจะสตรีมไฟล์เป็นส่วนเล็ก ๆ และติดตามความคืบหน้าของการอัปโหลดในเครื่อง ซึ่งกระบวนการดังกล่าวจะถูกจัดการโดย Tus client ตามแผนภาพ \autoref{fig:file upload sequence}
\fi

\insertPDFfigure[page=1,width=100mm]
{resources/mobile-app/file-upload-sequence.pdf}
{\ifenglish File Upload Sequence \else ลำดับการอัปโหลดไฟล์ \fi}
{file upload sequence}

\ifenglish
However, the Tus client only supports resumable uploads for a single file at a time. To handle uploads for an entire directory, the application employs a wrapper called the Directory Upload Client. This client iterates over each file in the directory, managing their individual upload processes while tracking which files are uploaded, in progress, or pending upload. Additionally, the client tracks the overall progress of the video session upload through the Progress File Store service, which provides the data necessary to display upload progress on the home page, as shown in \autoref{fig:directory upload sequence}.
\else
อย่างไรก็ตาม Tus client รองรับการอัปโหลดต่อเนื่องได้เพียงหนึ่งไฟล์ต่อครั้งเท่านั้น เพื่อให้สามารถอัปโหลดทุกไฟล์ในไดเรกทอรี แอปพลิเคชันใช้ตัวจัดการที่เรียกว่า Directory Upload Client ซึ่งจะทำการค้นหาแต่ละไฟล์ในไดเรกทอรี ควบคุมกระบวนการอัปโหลดของแต่ละไฟล์ พร้อมติดตามสถานะการอัปโหลดว่า ไฟล์ใดอัปโหลดแล้ว กำลังอัปโหลด หรือยังรอการอัปโหลด นอกจากนี้ยังติดตามความคืบหน้าทั้งหมดของเซสชันวิดีโอผ่านบริการ Progress File Store ซึ่งให้ข้อมูลที่จำเป็นสำหรับการแสดงผลความคืบหน้าบนหน้าแรกของแอป ตามแผนภาพ \autoref{fig:directory upload sequence}
\fi

\insertPDFfigure[page=1,width=100mm]
{resources/mobile-app/directory-upload-sequence.pdf}
{\ifenglish Directory Upload Sequence \else ลำดับการอัปโหลดไดเรกทอรี \fi}
{directory upload sequence}

\ifenglish
It is essential to understand that a WorkManager task operates independently of the main application once it is initialized. WorkManager utilizes its own memory space, separate from the main application, which prevents direct communication between the two.
\else
แต่ว่า WorkManager ทำงานแยกจากแอปพลิเคชันหลักเมื่อเริ่มต้นการทำงาน หมายความว่า WorkManager จะใช้พื้นที่หน่วยความจำของตัวเอง ซึ่งแยกออกจากแอปพลิเคชันหลัก และไม่สามารถสื่อสารกันโดยตรงได้
\fi

\ifenglish
However, the application's home page requires the upload progress to be displayed in real time. To address this, the Progress Isolate Manager is employed. This component retrieves upload progress data from the Progress File Store, where WorkManager tasks save progress information. The manager then communicates this progress to the home page.
\else
อย่างไรก็ตาม หน้าหลักของแอปพลิเคชันจะต้องแสดงความคืบหน้าของการอัปโหลดแบบเรียลไทม์ ระบบจึงจะใช้ Progress Isolate Manager เพื่อดึงข้อมูลความคืบหน้าจาก Progress File Store ซึ่งเป็นที่จัดเก็บข้อมูลคืบหน้าในการอัปโหลดของ WorkManager จากนั้นตัวจัดการดังกล่าวจะสื่อสารข้อมูลไปยังหน้าแรกของแอปพลิเคชัน
\fi

\ifenglish
One challenge arises because the Progress Isolate Manager is unaware of when the WorkManager task updates the progress file in storage. Consequently, it must poll the storage for updated progress data—a process that is computationally expensive and inefficient for the main application.
\else
ปัญหาหนึ่งที่เกิดขึ้นคือ Progress Isolate Manager ไม่สามารถรู้ได้ว่าเมื่อใดที่ WorkManager อัปเดตไฟล์ความคืบหน้าในที่เก็บข้อมูล ตัวจัดการดังกล่าวเลยจึงต้องดึงข้อมูลจากที่เก็บเป็นระยะ ซึ่งเป็นกระบวนการที่ใช้ทรัพยากรสูงและไม่มีประสิทธิภาพสำหรับแอปพลิเคชันหลัก
\fi

\ifenglish
To mitigate this inefficiency, the system leverages the concept of isolates. An isolate functions similarly to a thread but operates with its own dedicated memory, separate from the main thread (referred to as the main isolate). Unlike WorkManager tasks, isolates can communicate with one another, including with the main isolate, making them an effective solution for bridging the gap between WorkManager tasks and the main application, as shown in \autoref{fig:upload progress polling}.
\else
ระบบจึงใช้แนวคิดของ isolates ซึ่งมีการทำงานคล้ายกับเธรด (Thread)แต่ใช้หน่วยความจำของตัวเองที่แยกจากเธรดหลัก (เรียกว่า main isolate) แต่จะแตกต่างจากงานของ WorkManager isolates ที่ว่า isolates จะสามารถสื่อสารกันเองและกับ main isolate ได้ จึงเป็นช่องทางในการเชื่อมโยงระหว่างข้อมูลจาก WorkManager และแอปพลิเคชันหลักตามที่แสดงใน \autoref{fig:upload progress polling}
\fi

\insertPDFfigure
{resources/mobile-app/upload-progress-polling.pdf}
{\ifenglish Upload Progress Polling \else การดึงข้อมูลความคืบหน้าในการอัปโหลด \fi}
{upload progress polling}
