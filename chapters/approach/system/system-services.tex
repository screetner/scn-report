\subsection{\ifenglish Backend Services \else ระบบหลังบ้าน (Backend Service) \fi}
\ifenglish \else
ระบบหลังบ้านของโครงงานนี้ได้ถูกออกแบบให้เป็น microservice architecture ซึ่งจะประกอบด้วยหลาย ๆ เซอร์วิสทำงานด้วยกันไม่ว่างจะเป็น Main Service (เซอร์วิสหลังบ้านตัวหลัก), Log Service (เซอร์วิสที่ใช้จัดเก็บ Audit Logs ของการใช้งาน), Tusd Reuseable Upload (เซอร์วิสที่ใช้ในการอัพโหลดไฟล์ที่มีความสามารถในการอัพโหลดต่อจากเดิมถึงเมื่อการเชื่อต่ออินเทอร์เน็ตขัดข่องระหว่างการอัพโหลด)

โดยที่เซอร์วิสทั้งหมดจะถูก Deploy บน Docker ซึ่งจะมีการจัดแบ่ง Private Network ภายใน Docker ไว้ตามรูป \ref{fig:system-architecture} ซึ่งจะประกอบไปด้วยสองเน็ตเวิร์คหลัก ๆ คือ
\begin{itemize}
    \item Default : ทำหน้าเป็นเน็ตเวิร์คที่จะ Deploy เซอร์วิสทั่วไปที่จำเป็นต่อการทำงานของระบบเช่น Portainer, Nginx Reverse Proxy, Zero Trust Client และ Tusd Reuseable Upload
    \item scn-service : ทำหน้าเป็นเน็ตเวิร์คเฉพาะที่ใช้ Deploy microservice ซึ่งจะประกอบไปด้วยสองเซอร์วิสหลัก และ API Gateway ซึ่งทำหน้าที่ Route เส้นทางของการเรียกใช้งานแต่ละเซอร์วิสจากผู้ใช้
\end{itemize}

จากรูป \ref{fig:system-architecture} จะเห็นได้ว่าแต่ละเน็ตเวิร์คจะมีการแบ่งส่วนอยู๋ในกล่องของตัวเองนั้นหมายความว่า แต่ละเน็ตเวิร์คจะไม่สามารถสื่อสารกันโดยตรงได้จะต้องสื่อสารผ่านเส้นทางที่ได้แสดงไว้ตามรูป \ref{fig:system-architecture} เท่านั้น

จากที่กล่าวมาข้างต้น จะเห็นได้ว่าระบบนี้ได้รับการออกแบบให้ทำงานภายใต้ Private Network โดยสมบูรณ์ ซึ่งหมายความว่าจะไม่มีช่องทางให้ผู้ดูแลระบบสามารถเข้าถึงหรือจัดการระบบได้โดยตรง อย่างไรก็ตาม จากภาพ \ref{fig:system-architecture} แสดงให้เห็นถึงเซอร์วิสที่เรียกว่า Zero Trust Client ซึ่งทำหน้าที่เป็น VPN Server เพื่อให้ผู้ดูแลระบบสามารถเชื่อมต่อเข้าสู่ระบบเพื่อจัดการและบำรุงรักษาได้ตามความจำเป็น
\fi