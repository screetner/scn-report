
\subsection{\ifenglish User Useage \else ผู้ใช้งานระบบ \fi}
\ifenglish xxx 
\else 
ในส่วนของผู้ใช้งานจะถูกแบ่งออกเป็นสองกลุ่มผู้ใช้งานหลัก ๆ คือ กลุ่มผู้ใช้งานผ่านเว็บไซต์และกลุ่มผู้ใช้งานผ่านแอปพลิเคชัน ซึ่งการใช้งานในกลุ่มที่แตกต่างกันก็จะต้องมีการเรียกใช้เซอร์วิสที่แตกต่างกันเช่นกัน ทั้งนี้จากรูป \ref{fig:system-architecture} จะเห็นได้ว่าไม่ว่าจะเป็นผู้ใช้งานจากส่วนใดก็ตามที่จะเรียกใช้งานเซอร์วิสจะต้องเรียกใช้งานผ่าน Nginx Reverse Proxy ซึ่งทำหน้าที่เป็นตัวจัดการ Traffic ว่าควรจะนำ Request นั้นเรียกใช้งานในเซอร์วิสใดของระบบ

ส่วนของแอปพลิเคชันนั้นในขณะที่ต้องเรียนใช้งานเซอร์วิสก็จะสามารถทำได้โดยการส่ง Request เป็น REST API ผ่านตัว Nginx Reverse Proxy เพื่อใช้งานได้โดยตรง

สองของเว็บไซต์จากรูป \ref{fig:system-architecture} จะเห็นได้ว่าเว็บไซต์จะ Deploy อยู่บนเซอร์วิสหนึ่งที่ชื่อว่า Heroku ซึ่งเปรียบเสมือนผู้ช่วยที่จะคอยจัดการเรื่องการ Deploy เว็บไซต์ให้เรา โดยจะทำงานเมื่อมีการเปลี่ยนแปลงของ Source Code ของ Repository เว็บไซต์ ทำให้การ Deploy เว็บของระบบเป็นไปได้โดยง่ายโดยทำงานผ่านสิ่งที่เรียกว่า CI/CD (continuous integration continuous deployment) อีกทั้งการใช้งานระบบสามารถเข้าใช้งานได้จาก Domain name ที่ชื่อว่า \href{https://www.screetner.studio}{www.screetner.studio}
\fi