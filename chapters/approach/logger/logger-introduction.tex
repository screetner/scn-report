\ifenglish
\else
ระบบ Log ซึ่งเป็นส่วนหนึ่งของของกรอบสีแดงที่สามารถเห็นได้ในรูป \ref{fig:system-architecture} โดยในระบบนี้มีบทบาทสำคัญในการเก็บข้อมูลที่เกิดขึ้นในระบบ โดยระบบ Log นี้จะทำหน้าที่เก็บข้อมูลที่เกิดขึ้นในระบบ และใช้ในการตรวจสอบปัญหาที่เกิดขึ้นในระบบ รวมถึงใช้ในการวิเคราะห์ปัญหาที่เกิดขึ้นในระบบ โดยในตอนนี้จะมีเพียงแค่ในส่วนของการเก็บข้อมูลที่เกิดขึ้นในระบบเท่านั้น

ระบบ Log ประกอบด้วยหลายส่วนที่ทำงานร่วมกันเพื่อให้บริการต่าง ๆ แก่ผู้ใช้ ซึ่งรวมถึง:
\begin{itemize}
    \item ฐานข้อมูล: ใช้สำหรับเก็บข้อมูลทั้งหมด
    \item เซิร์ฟเวอร์: ทำหน้าที่ประมวลผลคำขอจากผู้ใช้
    \item API: เป็นตัวกลางในการสื่อสารระหว่างระบบต่าง ๆ
\end{itemize}

เทคโนโลยีที่ใช้ในระบบ Log นี้ประกอบด้วย:
\begin{itemize}
    \item ภาษาโปรแกรม: Typescript
    \item ฐานข้อมูล: MongoDB
    \item เครื่องมือ: Docker, Git
    \item เฟรมเวิร์ก: ElysiaJs
\end{itemize}

การทำงานของระบบ Log นี้จะเริ่มจากการรับข้อมูลที่เกิดขึ้นในระบบ ผ่านทาง API จากนั้น Server จะทำการเก็บข้อมูลลงในฐานข้อมูล แล้วส่งผลลัพธ์กลับไปยังผู้ใช้

นอกจากนี้ ระบบยังมีการเรียกใช้งานฟังก์ชัน cronJob ที่ถูกตั้งเวลาให้ทำงานทุกวันเพื่อทำการลบข้อมูล Log ที่มีอายุเกิน 7 วันออกจากฐานข้อมูล เพื่อให้ฐานข้อมูลมีขนาดเล็กลงและมีประสิทธิภาพในการทำงานสูงสุด
\fi